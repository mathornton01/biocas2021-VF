\documentclass[12pt,journal,compsoc]{IEEEtran}
\usepackage[cmex10]{amsmath}
\usepackage{hyperref}
\hyphenation{op-tical net-works semi-conduc-tor}


\begin{document}
\title{Fourier Power Spectra Software Filter Designs and Characteristic Learning}
\author{Micah~Thornton,~\IEEEmembership{Student Member,~IEEE,}
        and~Monnie~McGee,~\IEEEmembership{Member,~ASA,}% 
\IEEEcompsocitemizethanks{\IEEEcompsocthanksitem M. Thornton is with the Lyda Hill Department
of Bioinformatics, University of Texas Southwestern, Dallas,
TX, 75390.\protect\\
E-mail: \url{mailto:mathornton@smu.edu}
\IEEEcompsocthanksitem M. McGee is with the Department of Statistical Science at Southern Methodist University.}% 
\thanks{Manuscript received June ??, 2021}}%; revised July ??, 2021}}

\markboth{Eighth Biomedical Circuits and Systems Conference,~October~2021}%
{Thornton \MakeLowercase{\textit{et al.}}: Fourier Power Spectra No-Lag Matched Filters for Genomic Sequence Differentiation}
\IEEEtitleabstractindextext{%
\begin{abstract}
The abstract goes here.
\end{abstract}
\begin{IEEEkeywords}
Computer Society, IEEEtran, journal, \LaTeX, paper, template.
\end{IEEEkeywords}}

\maketitle

\IEEEdisplaynontitleabstractindextext

\IEEEpeerreviewmaketitle

\section{Introduction}
\label{sec:int}

\IEEEPARstart{G}{enomic} sequences in their base nature consist of radix four discrete-index signals, 
and contain essential information required to instantiate and replicate organisms, or non-living matter 
such as viruses and proteins.
These sequences are Radix-four as a repetitive series of four base nucleotides: adenine, cytosine, guanine, 
and thymine, in Deoxyribonucleic Acid (DNA) with thymine is replaced by uracil in Ribonucleic Acids 
(RNA) constitute the information conveyed by the signals. 
Notably this characterization considers the base form of the genomic sequences as a four-valued signal, 
some prior works have considered using the hydrophobicity of the translated amino acids in conjunction 
with spectral methods for the identification of attributes of a sequence. [CITATION] 
Other works have described the Methylation and other epigenetic marker profiles using a 
in this work we consider utilization of subsets of characteristic genomic Fourier power spectra which are
produced by 

\subsection{Prior Works}
\label{sec:pw}
Yin et al. suggested the use of Fourier coefficients, and the resultant power spectra for numerically summarizing the information from genomic sequences [CITATION].


\subsubsection{Subsubsection Heading Here}



\section{Methods \& Design}
\label{sec:meth} 

\section{Results} 
\label{sec:res}

The procedures described in th

\section{Conclusion}
\label{sec:conc}


\appendices

\section*{Acknowledgment}



\begin{thebibliography}{1}

\bibitem{IEEEhowto:kopka}
H.~Kopka and P.~W. Daly, \emph{A Guide to \LaTeX}, 3rd~ed.\hskip 1em plus
  0.5em minus 0.4em\relax Harlow, England: Addison-Wesley, 1999.

\end{thebibliography}


\begin{IEEEbiographynophoto}{Micah Thornton}
Is a fourth year Ph.D. candidate in a joint Southern Methodist University (SMU) and University of Texas Southwestern (UTSW) Biostatistics Program.  Micah holds a Bachelor's of Science (B.S) in Computer Engineering, a B.S. in Statistical Science, and a Master's of Science (M.S) in Computer Engineering from SMU. 
\end{IEEEbiographynophoto}

\begin{IEEEbiographynophoto}{Monnie McGee} 

\end{IEEEbiographynophoto}

\end{document}


